\documentclass[main.tex]{subfiles}
\begin{document}
    \chapter{Electricity \& Magnetism}
        \label{ch: Electricity and Magnetism}
        \thispagestyle{noheader}
        Here electricity and magnetism will be treated separately, though as may become apparent, the two are actually strongly linked.

        \section{Electrostatics}
            \label{sec: Electrostatics}
            Here we will deal with the rules which govern static (stationary) charged particles.

            \subsection{Charge}
                \label{subsec: Charge}
                We are all used to talking about positively or negatively charged things, but how do we quantify that?\\
                Like most things we have to give it a unit, in this case named after a very influential physicist in this area: Coulomb.

                Because it was historically hard to measure charge directly, this unit is based off the Amp (also named after a physicist Ampere) such that $1\,C = 1\,A \cdot 1\,s$

            \subsection{The Electric Field}
                \label{subsec: The Electric Field}

                The electric field $(\vect{E})$ is a vector field representing the force that would be exerted on a $1\,C$ charged particle.
\end{document}